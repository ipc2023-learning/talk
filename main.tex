%----------------------------------------------------------------------------------------
%	PACKAGES AND THEMES
%----------------------------------------------------------------------------------------
\documentclass[aspectratio=169,xcolor=dvipsnames]{beamer}
\usetheme{SimplePlus}

\usepackage{hyperref}
\usepackage{graphicx} % Allows including images
\usepackage{booktabs} % Allows the use of \toprule, \midrule and \bottomrule in tables


\usepackage{marvosym}
\usepackage{fontawesome}
\usepackage[ruled,vlined,linesnumbered]{algorithm2e}
\usepackage{amsmath}
\usepackage{amssymb}
\usepackage{amsthm}
\usepackage{arydshln}
\usepackage{listings}
\usepackage{subcaption}
\usepackage{tikz}
\usetikzlibrary{arrows,automata,shapes}
\newcommand{\tup}[1]{{\langle #1 \rangle}}
\usepackage{multirow}
%\usepackage{minted}
%\usepackage{svg}
%----------------------------------------------------------------------------------------
%	TITLE PAGE
%----------------------------------------------------------------------------------------

\title[short title]{IPC 2023 - Learning Track} % The short title appears at the bottom of every slide, the full title is only on the title page


\author[Allauthors] {Jendrik Seipp \& Javier Segovia-Aguas}

\date{} % Date, can be changed to a custom date





%----------------------------------------------------------------------------------------
%	PRESENTATION SLIDES
%----------------------------------------------------------------------------------------

\begin{document}
\begin{frame}
    % Print the title page as the first slide
    \titlepage

\end{frame}

\begin{frame}{Domains}
    \begin{itemize}
        \item STRIPS, action costs, types, negative preconditions
        \item Training: $\sim$99 ``easy'' instances
        \item Testing: 30 easy, 30 medium and 30 hard instances
    \end{itemize}
\end{frame}

\begin{frame}{Blocksworld}

    \begin{itemize}
        \item \textbf{Description}: the task is to map an initial configuration of towers of $n$ blocks in total, into a goal configuration where block towers are arranged in a different order
        \item[]
        \item \textbf{Hardness}: $2$-approximable
        %\item \textbf{Required features}: goal lookups.
        \item[]
        \item \textbf{Parameter ranges}:
        \begin{itemize}
            \item Easy: $n \in [5, 30]$
            \item Medium: $n \in [35, 150]$
            \item Hard: $n \in [160, 500]$
        \end{itemize}
    \end{itemize}

\end{frame}

\begin{frame}{Childsnack}

    \begin{itemize}
        \item \textbf{Description}: $c$ children, $a$ allergic, $t$ trays, $s$ sandwiches % ToDo: add description
        \item[]
        \item \textbf{Hardness}: optimal
        %\item \textbf{Required Features}: ...
        \item[]
        \item \textbf{Parameter ranges}:
        \begin{itemize}
            \item Easy: $c\in[4, 10]$, $a\in[0, 6]$, $t\in[1, 3]$, $s\in[4, 15]$
            \item Medium: $c\in[15, 40]$, $a\in[15, 25]$, $t\in[2, 5]$, $s\in[15, 60]$
            \item Hard: $c\in[50, 300]$, , $a\in[50, 150]$, $t\in[4, 10]$, $s\in[50, 450]$
        \end{itemize}
    \end{itemize}

\end{frame}


\begin{frame}{Ferry}

    \begin{itemize}
        \item \textbf{Description}: ...
        \item[]
        \item \textbf{Hardness}: ...
        %\item \textbf{Required Features}: ...
        \item[]
        \item \textbf{Parameter ranges}:
        \begin{itemize}
            \item Easy: ...
            \item Medium: ...
            \item Hard: ...
        \end{itemize}
    \end{itemize}

\end{frame}


\begin{frame}{Floortile}

    \begin{itemize}
        \item \textbf{Description}: ...
        \item[]
        \item \textbf{Hardness}: ...
        %\item \textbf{Required Features}: ...
        \item[]
        \item \textbf{Parameter ranges}:
        \begin{itemize}
            \item Easy: ...
            \item Medium: ...
            \item Hard: ...
        \end{itemize}
    \end{itemize}

\end{frame}


\begin{frame}{Miconic}

    \begin{itemize}
        \item \textbf{Description}: ...
        \item[]
        \item \textbf{Hardness}: ...
        %\item \textbf{Required Features}: ...
        \item[]
        \item \textbf{Parameter ranges}:
        \begin{itemize}
            \item Easy: ...
            \item Medium: ...
            \item Hard: ...
        \end{itemize}
    \end{itemize}

\end{frame}


\begin{frame}{Rovers}

    \begin{itemize}
        \item \textbf{Description}: ...
        \item[]
        \item \textbf{Hardness}: ...
        %\item \textbf{Required Features}: ...
        \item[]
        \item \textbf{Parameter ranges}:
        \begin{itemize}
            \item Easy: ...
            \item Medium: ...
            \item Hard: ...
        \end{itemize}
    \end{itemize}

\end{frame}


\begin{frame}{Satellite}

    \begin{itemize}
        \item \textbf{Description}: ...
        \item[]
        \item \textbf{Hardness}: ...
        %\item \textbf{Required Features}: ...
        \item[]
        \item \textbf{Parameter ranges}:
        \begin{itemize}
            \item Easy: ...
            \item Medium: ...
            \item Hard: ...
        \end{itemize}
    \end{itemize}

\end{frame}


\begin{frame}{Sokoban}

    \begin{itemize}
        \item \textbf{Description}: ...
        \item[]
        \item \textbf{Hardness}: ...
        %\item \textbf{Required Features}: ...
        \item[]
        \item \textbf{Parameter ranges}:
        \begin{itemize}
            \item Easy: ...
            \item Medium: ...
            \item Hard: ...
        \end{itemize}
    \end{itemize}

\end{frame}


\begin{frame}{Spanner}

    \begin{itemize}
        \item \textbf{Description}: ...
        \item[]
        \item \textbf{Hardness}: ...
        %\item \textbf{Required Features}: ...
        \item[]
        \item \textbf{Parameter ranges}:
        \begin{itemize}
            \item Easy: ...
            \item Medium: ...
            \item Hard: ...
        \end{itemize}
    \end{itemize}

\end{frame}


\begin{frame}{Transport}

    \begin{itemize}
        \item \textbf{Description}: ...
        \item[]
        \item \textbf{Hardness}: ...
        %\item \textbf{Required Features}: ...
        \item[]
        \item \textbf{Parameter ranges}:
        \begin{itemize}
            \item Easy: ...
            \item Medium: ...
            \item Hard: ...
        \end{itemize}
    \end{itemize}

\end{frame}

\begin{frame}{Setup}
    \begin{itemize}
        \item all computations done by organizers
        \item only bugfixes allowed after submission deadline
        \item submission two Apptainer files
        \item goal: reproducibility
    \end{itemize}

    \begin{block}{Learning}
        ./learn dk DOMAIN TASK1 TASK2 TASK3 ...
    \end{block}

    produces files \texttt{dk.1}, \texttt{dk.2}, etc.

    \begin{exampleblock}{Planning}
        ./plan dk.5 DOMAIN TASK plan
    \end{exampleblock}

    finds plans \texttt{plan.1}, \texttt{plan.2}, etc.
\end{frame}

\begin{frame}{Environment}
    Single-Core
    \begin{itemize}
        \item 1 CPU core (from an Intel Xeon Gold 6130 CPU), no GPU
        \item Limits \alert{training} per domain: 24 hours, 32 GiB
        \item Limits \alert{evaluation} per task: 30 minutes, 8 GiB
    \end{itemize}

    \bigskip
    Multi-Core canceled
\end{frame}

\begin{frame}{Metrics}
    \begin{itemize}
        \item Quality score: C*/C
        \item Bounds C* obtained with domain-specific solvers, IPC planners, LAMA (8h, 32 GiB)
        \item Agile score: 1 - log(T)/log(300)
    \end{itemize}
\end{frame}

\section{Submissions}

\begin{frame}{Baselines}
\begin{itemize}
\item
    \textbf{Fast Downward SMAC 2014}\\
    \emph{Jendrik Seipp, Silvan Sievers, Frank Hutter}\\
    Single Fast Downward
    configuration, optimized with SMAC for training tasks.
\item
    \textbf{Progressive Generalized Planner}\\
    \emph{Javier
    Segovia-Aguas, Sergio Jiménez, Laura Sebastiá, Anders Jonsson}\\
    Fixed configuration of PGP for the given training tasks.
\end{itemize}
\end{frame}

\begin{frame}{Participants 1/2}
\begin{itemize}
\item
    \textbf{ASNets 2023}\\
    \emph{Mingyu Hao, Ryan Wang, Sam Toyer, Felipe Trevizan, Sylvie
    Thiébaux, Lexing Xie}\\
    Action Schema Networks implemented in Tensorflow 2.
\item
    \textbf{GOFAI}\\
    \emph{Alvaro Torralba, Daniel Gnad}\\
    Good Old-Fashioned AI that learns
    how to partially ground tasks from a given domain.
\item
    \textbf{HUZAR}\\
    \emph{Piotr Rafal Gzubicki, Bartosz Piotr Lachowicz, Alvaro Torralba}\\
    Learn to distinguish between good and bad transitions by feeding
    problem description graphs into a GNN.
\end{itemize}
\end{frame}

\begin{frame}{Participants 2/2}
\begin{itemize}
\item
    \textbf{Muninn}\\
    \emph{Simon Ståhlberg, Blai Bonet, Hector Geffner}\\
    Learn Relational MPNNs for STRIPS.
\item
    \textbf{Novelty-based Progressive Generalized Planner}\\
    \emph{Chao Lei, Nir Lipovetzky, Krista A. Ehinger}\\
    Novelty-based generalized planner that
    prunes a newly generated planning program if its most frequent action
    repetition is greater than a given bound.
\item
    \textbf{Vanir}\\
    \emph{Dominik Drexler}\\
    Learn width-based hierarchical policies for
    polynomial domains.
\end{itemize}
\end{frame}


\end{document}
